\documentclass{rapport}
\usepackage{lipsum}
\title{Rapport UCL - Template} %Titre du fichier

\begin{document}

%----------- Informations du rapport ---------

\logo{Logo Esiea - noir.png}
\unif{UCLouvain}
\titre{Titre du rapport} %Titre du fichier .pdf
\cours{Nom du cours} %Nom du cours
\sujet{Sujet du rapport} %Nom du sujet

\enseignant{Prénom \textsc{Nom}} %Nom de l'enseignant

\eleves{Prénom \textsc{Nom} \\
		Prénom \textsc{Nom} \\ 
		Prénom \textsc{Nom} } %Nom des élèves

%----------- Initialisation -------------------
        
\fairemarges %Afficher les marges
\fairepagedegarde %Créer la page de garde
\tabledematieres %Créer la table de matières

%------------ Corps du rapport ----------------


\section{Première section} 

\lipsum[3-4]%Effacer cette ligne et écrire le texte souhaité

\subsection{Subsection}

\lipsum[3-4] %Effacer cette ligne et écrire le texte souhaité

\section{Deuxième section}

\lipsum[3-5] %Effacer cette ligne et écrire le texte souhaité

%------------- Commandes utiles ----------------

\section{Quelques commandes}

Voici quelques commandes utiles :

%------- Pour insérer et citer une équation --------------

\begin{equation} \label{eq: exemple}
\rho + \Delta = 42
\end{equation}

L'équation \ref{eq: exemple} est cité ici. 

% ------- Pour écrire des variables ----------------------

Pour écrire des variables dans le texte, il suffit de mettre le symbole \$ entre le texte souhaité comme : constante $\rho$. 


\end{document}