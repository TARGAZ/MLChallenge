\documentclass{rapport}
\usepackage{lipsum}
\title{TAROT_MONNIER_POINGT_PLANCHENAULT} %Titre du fichier

\begin{document}

%----------- Informations du rapport ---------

\logo{Logo Esiea - noir.png}
\titre{Rapport MLChallenge} %Titre du fichier .pdf
\cours{Machine Learning Challenge} %Nom du cours

\enseignant{Thibault \textsc{GEOFFROY}} %Nom de l'enseignant

\eleves{Bastien \textsc{TAROT} \\
		Raphael \textsc{MONNIER} \\ 
		Tanguy \textsc{POINGT} \\ 
        Allan \textsc{PLANCHENAULT} } %Nom des élèves

%----------- Initialisation -------------------
        
\fairemarges %Afficher les marges
\fairepagedegarde %Créer la page de garde
\tabledematieres %Créer la table de matières

%------------ Corps du rapport ----------------


\section{Introduction} 
\subsection{Contexte}

\cite{koBriefReviewFacial2018}
\cite{sariyanidiAutomaticAnalysisFacial2015}

\subsection{Problématique}

\subsection{Objectifs du Projet}

\subsection{Etat de l'art}

\section{Analyse des Données}
\subsection{Description des Données}
\subsection{Exploration des Données}

\section{Prétraitement et Extraction des Features}
\subsection{Prétraitement des Données}
\subsection{Extraction des Features}
\subsection{Modèles et Méthodologies}

\section{Modèle(s) choisi(s)}
\subsection{Paramétrage et Entraînement}
\subsection{Expérimentations}
\subsection{Résultats}

\section{Évaluation des Modèles}
\subsection{Analyse des Résultats}
\subsection{Discussion et Limites}

\section{Discussion des Résultats}
\subsection{Limites}
\subsection{Conclusion et Perspectives}

\section{Bibliographie / Références}

\bibliographystyle{ieeetr}
\bibliography{MLChallengeBiblio/MLChallengeBiblio.bib}

\end{document}