\documentclass{rapport}
\usepackage{lipsum}
\title{TAROT_MONNIER_POINGT_PLANCHENAULT} %Titre du fichier

\begin{document}

%----------- Informations du rapport ---------

\logo{Logo Esiea - noir.png}
\titre{Rapport MLChallenge} %Titre du fichier .pdf
\cours{Machine Learning Challenge} %Nom du cours

\enseignant{Thibault \textsc{GEOFFROY}} %Nom de l'enseignant

\eleves{Bastien \textsc{TAROT} \\
    Raphael \textsc{MONNIER} \\
    Tanguy \textsc{POINGT} \\
    Allan \textsc{PLANCHENAULT} } %Nom des élèves

%----------- Initialisation -------------------

\fairemarges %Afficher les marges
\fairepagedegarde %Créer la page de garde
\tabledematieres %Créer la table de matières

%------------ Corps du rapport ----------------

\section{Introduction} 

La reconnaissance des émotions est un domaine en pleine expansion qui suscite l'intérêt de nombreux chercheurs en raison des défis complexes qu'il pose. 
Malgré les avancées réalisées, ce problème reste encore partiellement résolu. 
Ce champ d'étude est vaste, mais pour ce projet, nous nous concentrerons sur les émotions dites de base selon le modèle d'Ekman : 
la joie, la colère, le dégoût, la tristesse, la peur et la surprise, en ajoutant également l'absence d'émotion, c'est-à-dire l'état « neutre ».\\

\subsection{Objectifs du Projet}
Pour développer un système de reconnaissance des émotions en utilisant une approche de machine learning, il est généralement recommandé de suivre un pipeline structuré.\\

Dans le cadre de ce projet, apres la mise en place des différents sujets d'introduction nous aborderons les grands thèmes suivants :\\

1.Prétraitement et Extraction des Features : 
Les étapes nécessaires pour préparer les données et extraire les caractéristiques pertinentes pour la reconnaissance des émotions.\\

2.Modèles Choisis : 
Les différents modèles de machine learning sélectionnés pour cette tâche et les raisons de leur choix.\\

3.Évaluation des Modèles :
Les méthodes et les métriques utilisées pour évaluer la performance des modèles de reconnaissance d émotions.\\

4.Discussion des Résultats : 
Une analyse critique des résultats obtenus, mettant en évidence les points forts et les limitations des approches adoptées.\\

Cette structure permettra de couvrir de manière exhaustive les aspects clés du développement d'un système de reconnaissance des émotions basé sur l'apprentissage automatique.

\subsection{Etat de l'art}

\section{Analyse des Données}
\subsection{Description des Données}

Le visage manifeste près de 2/3 des émotions chez un humain 
[mettre ref premier pdf].Les zones les plus démonstratives sont 
principalement situé au niveau des lèvres, des sourcils, des yeux mais 
également des yeux et du nez (malgré qu'ils sont moins significatif sur leurs
représentations [mettre ref premier pdf]). Les données sont représentées dans 
un CSV. Il y a au total 978 observations, toutes sous le même format. Chacune 
d'entre est représenté sur une même ligne et 138 colonnes. La première colonne 
est réservé pour l'ID de l'image. La deuxième colonne représente simplement le 
label de l'observation. Pour les 136 colonnes restantes, elles sont séparées 
en deux sous-parties. En effet, les 64 premières valeurs et les 64 dernières 
valeurs sont respectivement les coordonnées en X et en Y de chaque points 
encadrant le visage et plus précisément les zones mentionnées plus tôt.

\subsection{Exploration des Données}

\section{Prétraitement et Extraction des Features}
\subsection{Prétraitement des Données}
\subsection{Extraction des Features}
\subsection{Modèles et Méthodologies}

\section{Modèle(s) choisi(s)}
\subsection{Paramétrage et Entraînement}
\subsection{Expérimentations}
\subsection{Résultats}

\section{Évaluation des Modèles}
\subsection{Analyse des Résultats}
\subsection{Discussion et Limites}

\section{Discussion des Résultats}
\subsection{Limites}
\subsection{Conclusion et Perspectives}

\section{Bibliographie / Références}

\bibliographystyle{ieeetr}
\bibliography{MLChallengeBiblio/MLChallengeBiblio.bib}

\end{document}